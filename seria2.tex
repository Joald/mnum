\documentclass{article}
\usepackage[utf8]{inputenc}
\usepackage{amsfonts}
\usepackage{polski}
\usepackage{textcomp}
\usepackage{color}
\usepackage{enumitem}
\usepackage{amsmath}
\usepackage{amssymb}
\usepackage{mathtools}
\usepackage{listings}
\usepackage{amsthm}
\usepackage{amsmath}
\usepackage{fancyhdr}
\pagestyle{fancy}


\definecolor{dkgreen}{rgb}{0,0.6,0}
\definecolor{gray}{rgb}{0.5,0.5,0.5}
\definecolor{mauve}{rgb}{0.58,0,0.82}
\lstset{frame=tb,
	language=Bash,
	aboveskip=3mm,
	belowskip=3mm,
	showstringspaces=false,
	columns=flexible,
	basicstyle={\small\ttfamily},
	numbers=none,
	numberstyle=\tiny\color{gray},
	keywordstyle=\color{blue},
	commentstyle=\color{dkgreen},
	stringstyle=\color{mauve},
	breaklines=true,
	breakatwhitespace=true,
	escapeinside={(*}{*)},
	tabsize=4
}
\cfoot{Jacek Olczyk}
\title{Seria 2}
\author{Jacek Olczyk}
\begin{document}
\maketitle
\begin{enumerate}
\item[5.] Funkcję $f(x)=\cos(2x)$ interpolujemy na przedziale $[-1,4]$ w punktach $ \{-1, 1, 2, 3, 4\} $ funkcją $ w $ będącą wielomianem liniowym na $ [-1,1] $ oraz wielomianem stopnia co najwyżej 3 na $ [1,4] $. Czy taka funkcja $ w $ jest wyznaczona jednoznacznie? Oszacuj możliwie dokładnie błąd interpolacji $ e=\|f-w\|_{\infty, [-1, 4]} $.

\begin{proof}[Rozwiązanie]
Tak, taka funkcja jest wyznaczona jednoznacznie, ponieważ każdy z wielomianów jest wyznaczony jednoznacznie, jako że wiadomo z wykładu że interpolacja Lagrange'a w $ n + 1 $ węzłach wielomianem stopnia $ n $ zawsze istnieje i jest jednoznaczna. Na przedziale $ [-1,1] $ wielomian jest liniowy, a jako że cosinus jest parzysty to w obu węzłach jest ta sama wartość, zatem na całym tym przedziale $ w(x)=\cos2<0 $. Zatem skoro $ \cos0=1 $, to na tym przedziale błąd interpolacji wynosi $ 1 - cos(2) $. Na przedziale $ [1,4] $ możemy ograniczyć z góry błąd używając wzorów z ćwiczeń: 
\begin{align*}
\|f-w\|_{\infty, [1,4]} &\le 
\frac{
	\|f^{(n+1)}\|_{\infty,[1,4]}
}{
	(n+1)!
} \|p\|_{\infty,[1,4]}\\
&\le \frac{
	\|f^{(4)}\|_{\infty,[1,4]}
}{
	4!
} \|p\|_{\infty,[1,4]}\\
&\le\frac{ \|16\cos(2x)\|}{24}\|p\|_{\infty,[1,4]}\\
&\le \frac{2}{3}\|p\|_{\infty,[1,4]}\\
&\le \frac{2}{3} \cdot \frac{3!}{4}\\
&\le 1 
\end{align*}
Jako że $ 1 < 1 - \cos2 $, to $ 1 - \cos2 $ jest dokładnym górnym ograniczeniem błędu interpolacji.
\end{proof}
\newpage
\item [7.] Wyznacz wielomian $ w $ stopnia co najwyżej 2 optymalny dla funkcji $ f(x)=x^3 $ w sensie aproksymacji średniokwadratowej w normie $ \|g\|=\sqrt{(g, g)} $ zadanej przez iloczyn skalarny $ (f, g)=\int_{0}^{1}f(x)g(x)xdx $. Oblicz $ \|f-w\| $.
\begin{proof}[Rozwiązanie]
	Baza = $ \{1, x, x^2\} $.
	\begin{align*}
		G&=
		\begin{bmatrix}
		\int_{0}^{1}xdx & \int_{0}^{1}x^2dx & \int_{0}^{1}x^3dx\\
		\int_{0}^{1}x^2dx & \int_{0}^{1}x^3dx & \int_{0}^{1}x^4dx \\
		\int_{0}^{1}x^3dx & \int_{0}^{1}x^4dx & \int_{0}^{1}x^5dx 
		\end{bmatrix} =
		\begin{bmatrix}
			\frac12&\frac13&\frac14\\
			\frac13&\frac14&\frac15\\
			\frac14&\frac15&\frac16
		\end{bmatrix} &
		F=	\begin{bmatrix}
			\frac15\\ \frac16\\ \frac17
		\end{bmatrix}\\
		Gc&=F & c =
		\begin{bmatrix}
			\frac4{35}\\
			\frac67\\
			\frac{12}7
		\end{bmatrix}
	\end{align*}
	\begin{align*}
		w(x)&=\frac4{35}- \frac67x + \frac{12}7x^2 \\
		w(x) - f(x) &= \frac4{35}- \frac67x + \frac{12}7x^2 - x^3\\
		\|w-f\|&=\sqrt{\int_{0}^{1}(w(x)-f(x))^2xdx}=\\
		&=  \sqrt{\int_{0}^{1}\frac{16}{1225} - \frac{48}{245}x^2 + \frac{276}{245}x^3 - \frac{776}{245}x^4 + \frac{228}{49}x^5 - \frac{24}{7}x^6 + x^7dx}=\\
		&=\sqrt{\frac1{9800}}=\frac{\sqrt2}{140}
	\end{align*}
	
\end{proof}
\end{enumerate}
\end{document}